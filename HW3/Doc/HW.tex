% !TEX program = xelatex
\documentclass[]{article}
\usepackage{commons/course}

\begin{document}
\printheader

در این داکیومنت به تحلیل کدی که برای قسمت سوال زدیم می‌پردازیم.
در ابتدا چندین کلاس تعریف شده‌اند که نشان‌دهنده‌ی هر نوع تسک پریودیک یا آپریودیک هستند.
هر تسک یک
\verb|id|
نیز دارد که صرفا برای نشان دادن آن استفاده می‌شود.
هر تسک آپریودیک یک زمان ریلیز دارد و همچنین یک عدد دارد که نشان می‌دهد که چه قدر دیگر مانده
است که کارش تمام شود. در ابتدا این عدد برابر همان
$C$
است. برای تسک‌های پریودیک نیز همین متغیر‌ها تعریف می‌شوند فقط یک متغیر دیگر به اسم
\verb|interval|
نیز داریم که نشان می‌دهد که پریود این تسک چه قدر است.

در ادامه یک کلاس به اسم
\lr{PoolingServer}
تعریف می‌کنیم که عملا همان کلاسی است که تمامی مشخصات سرور را دارد. یک تابع نیز برای این
سرور تعریف می‌کنیم که نشان می‌دهد که آیا سیستم قابل زمانبندی است یا خیر که به کمک
فرمول اسلاید ۵ صفحه‌ی ۱۳ کار می‌کند.

در ادامه نیز یک تابع تعریف می‌کنیم که صرفا فایل جیسون ورودی برنامه را می‌خواند و آن را تبدیل
به یک
\lr{PoolingServer}
می‌کند.

در نهایت نیز تابع زمانبندی خودمان را می‌نویسیم. سعی می‌کنم که در اینجا کاری که کردم را به صورت خلاصه
توضیح دهم. در ابتدا متغیری تعریف شده است به اسم
\lr{handling aperiodic tasks}
که در صورتی که برابر
\verb|True|
باشد نشان می‌دهد که درحال حاضر در حال انجام دادن تسک‌های آپریودیک هستیم و آن‌ها می‌توانند که
تسک‌های پریودیک را
\lr{preempt}
بکنند. در ادامه متغیر
\lr{current time}
را که زمان فعلی شبیه سازی را نشان می‌دهد را از یک تا زمان مورد نظر جلو می‌بریم.
در هر لحظه از زمان در ابتدا در صورتی که زمان پریود یکی از تسک‌ها بود زمان
\lr{remaining computaion}
آن را برابر
$C$
قرار می‌دهیم و با این کار عملا مشخص می‌کنیم که این تسک می‌تواند زمان بندی شود.
در ادامه در صورتی که در
\lr{interval}
خود سرور بودیم نیز چک می‌کنیم که آیا اصلا تسک آپریودیکی وجود دارد که قابل زمانبندی باشد
یا خیر. در این صورت متغیر
\lr{handling aperiodic tasks}
را برابر
\verb|True|
قرار می‌دهیم و مقدار
\lr{server capacity}
مانده را همان
\lr{server capacity}
اولیه قرار می‌دهیم که تسک‌ها از آن استفاده کنند.

حال زمانبندی واقعی را شروع می‌کنیم. در ابتدا در صورتی که تسک‌هایی با پریود کمتر از
\lr{server}
وجود داشته باشند که زمان بندی نشده باشد به خاطر
\lr{RM}
بودن سرور آن‌ها را زمان بندی می‌کنیم. در ادامه سراغ تسک‌های
آپریودیک می‌رویم در صورتی که متغیر
\lr{handling aperiodic tasks}
\verb|True|
باشد. در صورتی که تسک‌های آپریودیک تمام شده بودند هم این متغیر را
\verb|False|
می‌کنیم.
در ادامه نیز در صورتی که هیچ چیزی تا اینجا زمان بندی نشده بود تسک‌هایی با پریود بیشتر از سرور
را زمانبندی می‌کنیم.

در خود بدنه‌ی اصلی اسکریپت نیز اول چک می‌کنیم که آیا تسک‌ها قابلیت زمان بندی شدن را دارند یا خیر.
در صورتی که قابلیت زمان بندی شدن را داشتند آن‌ها را زمانبندی می‌کنیم و در غیر این صورت کلا از برنامه
خارج می‌شویم.

برای مثال نیز من دقیقا مثال اسلاید‌ها را آوردم. اما با این اعداد تست زمانبندی فیل می‌شود چرا
که کلا این فرمول بد بینانه است. با این حال من زمان بندی را گفتم که انجام بده و نتیجه به صورت
زیر است:
\begin{latin}
\begin{lstlisting}
Did one unit of periodic task 5 in T=0
Did one unit of periodic task 6 in T=1
Did one unit of periodic task 6 in T=2
Did one unit of periodic task 5 in T=4
Starting to handle aperiodic tasks in T=5
Did one unit of aperiodic task 1 in T=5. Server has 1 units of capacity left
Did one unit of aperiodic task 1 in T=6. Server has 0 units of capacity left
Did one unit of periodic task 6 in T=7
Did one unit of periodic task 5 in T=8
Did one unit of periodic task 6 in T=9
Starting to handle aperiodic tasks in T=10
Did one unit of aperiodic task 2 in T=10. Server has 1 units of capacity left
Did one unit of periodic task 5 in T=12
Did one unit of periodic task 6 in T=13
Did one unit of periodic task 6 in T=14
Starting to handle aperiodic tasks in T=15
Did one unit of aperiodic task 3 in T=15. Server has 1 units of capacity left
Did one unit of periodic task 5 in T=16
Did one unit of aperiodic task 3 in T=17. Server has 0 units of capacity left
Did one unit of periodic task 6 in T=18
Did one unit of periodic task 6 in T=19
Starting to handle aperiodic tasks in T=20
Did one unit of periodic task 5 in T=20
Did one unit of aperiodic task 4 in T=21. Server has 1 units of capacity left
Did one unit of periodic task 5 in T=24
\end{lstlisting}
\end{latin}
همان طور که مشاهده می‌کنید بازه‌های زمان بندی مانند اسلاید است.
\end{document}
