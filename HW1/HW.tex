% !TEX program = xelatex
\documentclass[]{article}
\usepackage{commons/course}

\begin{document}
\printheader

راهی شبیه به اثبات
\lr{optimal}
بودن
\lr{EDD}
پیش می‌گیریم. در ابتدا مشخص است که که اگر هر دو تسک در یک زمان آمده باشند عملا زمانبند ما مثل
\lr{EDD}
می‌شود و می‌دانیم که
\lr{EDD optimal}
است. در حالت بعدی فرض کنید که تسک با ددلاین زودتر، زودتر از تسکی با ددلاین دیرتر آمده باشد.
فرض می‌کنیم که زمان اجرای تسکی که ددلاینش زودتر است نیز زودتر شروع شده است.
در اینجا می‌توان دو کار کرد. یا مثل الگوریتم
\lr{EDF}
زمانی که تسک جدید می‌آید، از آنجا که ددلاینش بعد از تسک فعلی است آن را
\lr{preempt}
نمی‌کنیم یا اینکه تسک جدید را در
\lr{CPU}
قرار می‌دهیم و تسک قبلی را
\lr{preempt}
می‌کنیم. فرض کنید که تسک با ددلاین زودتر
$A$
باشد و آن یکی $B$ باشد. آنگاه برای زمانبندی
\lr{EDF}
داریم:
\begin{align*}
    d_A &< d_B\\
    f_A &< f_B\\
    L_{\text{max}} &= \max(L_{A}, L_{B}) = \max(f_A - d_A, f_B - d_B)
\end{align*}
اما برای دومین زمانبندی که معرفی کردیم قطعا
\lr{lateness}
$A$
بیشتر است چرا که به ددلاین خودش نزدیک تر است نسبت به تسک
$B$.
پس داریم:
\begin{align*}
    d_A &< d_B\\
    f'_A &> f'_B\\
    L'_{\text{max}} &= \max(L'_{A}, L'_{B}) = L'_{A} = f'_A - d_A
\end{align*}
حال دو حالت را در نظر می‌گیریم. زمانی که
$\max(L_{A}, L_{B}) = f_A - d_A$
باشد و زمانی که
$\max(L_{A}, L_{B}) = f_B - d_B$
باشد. در حالت اول از آن‌جا که
$f_A < f'_A$
است پس در نتیجه
$L'_{\text{max}} > L_{\text{max}}$
است که این نشان می‌دهد زمانبند ما در یک حالت بدتر عمل کرده است و مورد قبول نیست!

حال بررسی می‌کنیم که یک تسک با ددلاین زودتر بعد از یک تسک با ددلاین دیرتر آمده است. در این حال ما باید
\lr{preemptation}
انجام دهیم در الگوریتم
\lr{EDF}. حال فرض کنید که الگوریتمی وجود دارد که
\lr{preemptation}
انجام نمی‌دهد و به اجرای تسک قبلی می‌پردازد. فرض کنید که تسکی که از قبل در پردازنده وجود داشت که ددلاین آن دیرتر است
$A$ است و آن یکی تسک $B$ است.
در این حالت برای الگوریتم
\lr{EDF}
داریم:
\begin{align*}
    d_A &> d_B\\
    f_A &> f_B\\
    L_{\text{max}} &= \max(L_{A}, L_{B}) = \max(f_A - d_A, f_B - d_B)
\end{align*}
همچنین برای آن یکی زمانبند داریم:
\begin{align*}
    d_A &> d_B\\
    f'_A &< f'_B\\
    L'_{\text{max}} &= \max(L'_{A}, L'_{B}) = L'_B = f'_B - d_B
\end{align*}
در این حالت نیز مانند قسمت قبل دو حالت می‌کنیم. در حالتی که
$L_{A} < L_{B}$
است داریم:
$L_{\text{max}} = f_B - d_B$ و $L'_{\text{max}} = f'_B - d_B$
است. از آنجا که
$f'_B > f_B$
است پس
$L_{\text{max}} < L'_{\text{max}}$
است. پس در نتیجه این الگوریتم
\lr{lateness}
را بیشتر کرده است.
\end{document}
